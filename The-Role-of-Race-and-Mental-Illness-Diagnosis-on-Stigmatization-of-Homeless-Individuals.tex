\documentclass[
  man,
  floatsintext,
  longtable,
  nolmodern,
  notxfonts,
  notimes,
  colorlinks=true,linkcolor=blue,citecolor=blue,urlcolor=blue]{apa7}

\usepackage{amsmath}
\usepackage{amssymb}



\usepackage[bidi=default]{babel}
\babelprovide[main,import]{english}


\babelfont{rm}[,RawFeature={fallback=mainfontfallback}]{Times New Roman}
% get rid of language-specific shorthands (see #6817):
\let\LanguageShortHands\languageshorthands
\def\languageshorthands#1{}

\RequirePackage{longtable}
\RequirePackage{threeparttablex}

\makeatletter
\renewcommand{\paragraph}{\@startsection{paragraph}{4}{\parindent}%
	{0\baselineskip \@plus 0.2ex \@minus 0.2ex}%
	{-.5em}%
	{\normalfont\normalsize\bfseries\typesectitle}}

\renewcommand{\subparagraph}[1]{\@startsection{subparagraph}{5}{0.5em}%
	{0\baselineskip \@plus 0.2ex \@minus 0.2ex}%
	{-\z@\relax}%
	{\normalfont\normalsize\bfseries\itshape\hspace{\parindent}{#1}\textit{\addperi}}{\relax}}
\makeatother




\usepackage{longtable, booktabs, multirow, multicol, colortbl, hhline, caption, array, float, xpatch}
\setcounter{topnumber}{2}
\setcounter{bottomnumber}{2}
\setcounter{totalnumber}{4}
\renewcommand{\topfraction}{0.85}
\renewcommand{\bottomfraction}{0.85}
\renewcommand{\textfraction}{0.15}
\renewcommand{\floatpagefraction}{0.7}

\usepackage{tcolorbox}
\tcbuselibrary{listings,theorems, breakable, skins}
\usepackage{fontawesome5}

\definecolor{quarto-callout-color}{HTML}{909090}
\definecolor{quarto-callout-note-color}{HTML}{0758E5}
\definecolor{quarto-callout-important-color}{HTML}{CC1914}
\definecolor{quarto-callout-warning-color}{HTML}{EB9113}
\definecolor{quarto-callout-tip-color}{HTML}{00A047}
\definecolor{quarto-callout-caution-color}{HTML}{FC5300}
\definecolor{quarto-callout-color-frame}{HTML}{ACACAC}
\definecolor{quarto-callout-note-color-frame}{HTML}{4582EC}
\definecolor{quarto-callout-important-color-frame}{HTML}{D9534F}
\definecolor{quarto-callout-warning-color-frame}{HTML}{F0AD4E}
\definecolor{quarto-callout-tip-color-frame}{HTML}{02B875}
\definecolor{quarto-callout-caution-color-frame}{HTML}{FD7E14}

%\newlength\Oldarrayrulewidth
%\newlength\Oldtabcolsep


\usepackage{hyperref}




\providecommand{\tightlist}{%
  \setlength{\itemsep}{0pt}\setlength{\parskip}{0pt}}
\usepackage{longtable,booktabs,array}
\usepackage{calc} % for calculating minipage widths
% Correct order of tables after \paragraph or \subparagraph
\usepackage{etoolbox}
\makeatletter
\patchcmd\longtable{\par}{\if@noskipsec\mbox{}\fi\par}{}{}
\makeatother
% Allow footnotes in longtable head/foot
\IfFileExists{footnotehyper.sty}{\usepackage{footnotehyper}}{\usepackage{footnote}}
\makesavenoteenv{longtable}

\usepackage{graphicx}
\makeatletter
\newsavebox\pandoc@box
\newcommand*\pandocbounded[1]{% scales image to fit in text height/width
  \sbox\pandoc@box{#1}%
  \Gscale@div\@tempa{\textheight}{\dimexpr\ht\pandoc@box+\dp\pandoc@box\relax}%
  \Gscale@div\@tempb{\linewidth}{\wd\pandoc@box}%
  \ifdim\@tempb\p@<\@tempa\p@\let\@tempa\@tempb\fi% select the smaller of both
  \ifdim\@tempa\p@<\p@\scalebox{\@tempa}{\usebox\pandoc@box}%
  \else\usebox{\pandoc@box}%
  \fi%
}
% Set default figure placement to htbp
\def\fps@figure{htbp}
\makeatother


% definitions for citeproc citations
\NewDocumentCommand\citeproctext{}{}
\NewDocumentCommand\citeproc{mm}{%
  \begingroup\def\citeproctext{#2}\cite{#1}\endgroup}
\makeatletter
 % allow citations to break across lines
 \let\@cite@ofmt\@firstofone
 % avoid brackets around text for \cite:
 \def\@biblabel#1{}
 \def\@cite#1#2{{#1\if@tempswa , #2\fi}}
\makeatother
\newlength{\cslhangindent}
\setlength{\cslhangindent}{1.5em}
\newlength{\csllabelwidth}
\setlength{\csllabelwidth}{3em}
\newenvironment{CSLReferences}[2] % #1 hanging-indent, #2 entry-spacing
 {\begin{list}{}{%
  \setlength{\itemindent}{0pt}
  \setlength{\leftmargin}{0pt}
  \setlength{\parsep}{0pt}
  % turn on hanging indent if param 1 is 1
  \ifodd #1
   \setlength{\leftmargin}{\cslhangindent}
   \setlength{\itemindent}{-1\cslhangindent}
  \fi
  % set entry spacing
  \setlength{\itemsep}{#2\baselineskip}}}
 {\end{list}}
\usepackage{calc}
\newcommand{\CSLBlock}[1]{\hfill\break\parbox[t]{\linewidth}{\strut\ignorespaces#1\strut}}
\newcommand{\CSLLeftMargin}[1]{\parbox[t]{\csllabelwidth}{\strut#1\strut}}
\newcommand{\CSLRightInline}[1]{\parbox[t]{\linewidth - \csllabelwidth}{\strut#1\strut}}
\newcommand{\CSLIndent}[1]{\hspace{\cslhangindent}#1}





\usepackage{fontspec} 

\defaultfontfeatures{Scale=MatchLowercase}
\defaultfontfeatures[\rmfamily]{Ligatures=TeX,Scale=1}

  \setmainfont[,RawFeature={fallback=mainfontfallback}]{Times New Roman}




\title{The Role of Race and Mental Illness Diagnosis on Stigmatization
of Homeless Individuals}


\shorttitle{STIGMA, HOMELESSNESS, MENTAL ILLNESS, AND RACE}


\usepackage{etoolbox}






\author{Karen Veronica Becerra}



\affiliation{
{Department of Psychology, The University of Chicago}}




\leftheader{Becerra}



\abstract{Homelessness in the United States is a persistent problem that
can have serious implications on the well-being of homeless individuals.
The present study focused on the role of race and mental illness
diagnosis on the stigmatization of homeless individuals, specifically
looking at the outcomes of the Attribution Questionnaire. This
questionnaire assessed the aspects of social distance, blame,
dangerousness, concern, and willingness to help of 215 participants
varying in ages across adulthood. The study was a self-paced online form
that used six experimental vignettes. The results indicated that there
were no significant interactions of race x diagnosis on stigmatization.
Additionally, race had no significant main effects, suggesting it was
not a significant factor for stigmatization of homeless individuals.
However, there were some significant main effects of diagnosis. Findings
might suggest that future work in reducing mental illness stigma and
increasing education could help decrease stigmatization of the homeless
population. }

\keywords{Homelessness, Stigmatization, Race, Mental
Illness, Diagnosis, Attribution Questionnaire}

\authornote{\par{\addORCIDlink{Karen Veronica
Becerra}{0009-0006-4967-0955}} 

\par{       }
\par{Correspondence concerning this article should be addressed to Karen
Veronica Becerra, Department of Psychology, The University of
Chicago, 5848 S. University
Avenue, Chicago, IL 60637, USA, Email: kvbecerra@uchicago.edu}
}

\makeatletter
\let\endoldlt\endlongtable
\def\endlongtable{
\hline
\endoldlt
}
\makeatother

\urlstyle{same}



\makeatletter
\@ifpackageloaded{caption}{}{\usepackage{caption}}
\AtBeginDocument{%
\ifdefined\contentsname
  \renewcommand*\contentsname{Table of contents}
\else
  \newcommand\contentsname{Table of contents}
\fi
\ifdefined\listfigurename
  \renewcommand*\listfigurename{List of Figures}
\else
  \newcommand\listfigurename{List of Figures}
\fi
\ifdefined\listtablename
  \renewcommand*\listtablename{List of Tables}
\else
  \newcommand\listtablename{List of Tables}
\fi
\ifdefined\figurename
  \renewcommand*\figurename{Figure}
\else
  \newcommand\figurename{Figure}
\fi
\ifdefined\tablename
  \renewcommand*\tablename{Table}
\else
  \newcommand\tablename{Table}
\fi
}
\@ifpackageloaded{float}{}{\usepackage{float}}
\floatstyle{ruled}
\@ifundefined{c@chapter}{\newfloat{codelisting}{h}{lop}}{\newfloat{codelisting}{h}{lop}[chapter]}
\floatname{codelisting}{Listing}
\newcommand*\listoflistings{\listof{codelisting}{List of Listings}}
\makeatother
\makeatletter
\makeatother
\makeatletter
\@ifpackageloaded{caption}{}{\usepackage{caption}}
\@ifpackageloaded{subcaption}{}{\usepackage{subcaption}}
\makeatother

% From https://tex.stackexchange.com/a/645996/211326
%%% apa7 doesn't want to add appendix section titles in the toc
%%% let's make it do it
\makeatletter
\xpatchcmd{\appendix}
  {\par}
  {\addcontentsline{toc}{section}{\@currentlabelname}\par}
  {}{}
\makeatother

%% Disable longtable counter
%% https://tex.stackexchange.com/a/248395/211326

\usepackage{etoolbox}

\makeatletter
\patchcmd{\LT@caption}
  {\bgroup}
  {\bgroup\global\LTpatch@captiontrue}
  {}{}
\patchcmd{\longtable}
  {\par}
  {\par\global\LTpatch@captionfalse}
  {}{}
\apptocmd{\endlongtable}
  {\ifLTpatch@caption\else\addtocounter{table}{-1}\fi}
  {}{}
\newif\ifLTpatch@caption
\makeatother

\begin{document}

\maketitle


\setcounter{secnumdepth}{-\maxdimen} % remove section numbering

\setlength\LTleft{0pt}


Homelessness in the United States and the struggle to give individuals
adequate housing is a persistent problem. Before the Covid -19 pandemic,
the number of homeless individuals was on the rise with 568,000
individuals experiencing homelessness in 2019, an increase of 15,000
from the previous
year(\citeproc{ref-frostHomelessnessWasRise2020}{Frost,
2020})\footnote{Frost
  (\citeproc{ref-frostHomelessnessWasRise2020}{2020}) provided
  information for the number of homeless individuals during the COVID 19
  pandemic a more current report can be found at
  https://www.statista.com/statistics/727847/homelessness-rate-in-the-us-by-state/}.
With the current Covid-19 pandemic, we can only predict that those
numbers have continued to increase. In the United States, 2.4\% of
homeless individuals die each year (Stasha, 2020). We know that the
general population often tries to distance itself from the stigmatized
population, more specifically the homeless population. Homeless
individuals face greater stigma and social isolation and often are
removed from public parks and other locations because the general public
does not want them too close. The problems caused by stigmatization,
such as social distancing, can affect the homeless population in terms
of resources that they have available such as sanitation centers,
employment, and social support. Often the homeless population lacks
resources and is exposed to the elements which can increase their
mortality, as well as the chance of being malnourished, having parasitic
infestations, periodontal disease, degenerative joint diseases, venereal
diseases, cirrhosis, and hepatitis-related to intravenous (IV) drug
abuse. Public attitudes toward homeless individuals can influence
policies and the services provided to this population. The attitudes
displayed through the stigma of homeless individuals can have an impact
on both physical and psychological health and willingness to access
services. The impact of these stigmas has shown to have serious
implications on the well-being of homeless individuals. The present
study examined factors that could predict levels of stigmatization
expressed towards homeless individuals.

\subsection{Literature review}\label{literature-review}

Research on the stigma of mental illness, homelessness, and race
highlights its harmful effects on health and social integration. P. W.
Corrigan et al. (\citeproc{ref-corriganPublicStigmaMental2009}{2009})
examined public stigma, focusing on stereotypes like causal attribution
(blaming individuals for their condition) and dangerousness (perceiving
them as threatening). Using vignette-based experiments, the study found
that people with psychiatric disorders, especially those with drug
addiction, faced greater stigma than those with physical disabilities,
laying the foundation for understanding how schizophrenia and substance
use disorders contribute to homelessness stigma.

While P. Corrigan et al.
(\citeproc{ref-corriganAttributionModelPublic2003}{2003}) explored
mental illness stigma, it did not examine health outcomes. In contrast,
Weisz and Quinn
(\citeproc{ref-weiszStigmatizedIdentitiesPsychological2018}{2018})
demonstrated that homelessness stigma leads to psychological distress,
poor health, and social avoidance. Among 175 volunteers at a homeless
event, those experiencing or anticipating stigma reported worse physical
and mental health and greater reluctance to seek services. Participants
of color faced even higher distress and service avoidance, highlighting
the compounded impact of race and homelessness stigma.

Building on this, Markowitz and Syverson
(\citeproc{ref-markowitzRaceGenderHomelessness2021}{2021}) investigated
race and gender intersections in stigma. They found that black homeless
individuals were perceived as more dangerous than white counterparts,
though no significant differences in social distance emerged. However,
the study's reliance on college-aged participants, who may have been
more tolerant than the general population, was a limitation. The present
study addresses this by including a broader, more diverse sample.

Similarly, Gattis and Larson
(\citeproc{ref-gattisPerceivedRacialSexual2016}{2016}) linked racial
stigma and discrimination to heightened depression among 89 black
adolescents and young adults experiencing homelessness. Using social and
minority stress models, the study highlighted how marginalized groups
face greater psychological distress due to limited societal support.
Though constrained by a small sample, it reinforced the role of racial
stigma in homelessness experiences.

In sum, stigma related to homelessness, mental illness, and race
profoundly affects psychological and physical health, social
integration, and resource access. The present study expands on this
research by examining how mental illness and race interact to shape
stigma, offering a more nuanced understanding of its impact on homeless
individuals

\subsection{Current Study}\label{current-study}

Building on the research by P. W. Corrigan et al.
(\citeproc{ref-corriganPublicStigmaMental2009}{2009}), Markowitz and
Syverson (\citeproc{ref-markowitzRaceGenderHomelessness2021}{2021}), and
Weisz and Quinn
(\citeproc{ref-weiszStigmatizedIdentitiesPsychological2018}{2018}) the
current study aimed to explore how race and mental illness diagnosis
impact the stigmatization of homeless individuals. The research
specifically focused on mental illness, distinguishing between
individuals with schizophrenia and those with substance use disorders,
and examined how these factors interact with race in shaping stigma.
Previous studies suggest that public stigma varies across mental health
conditions and that race plays a crucial role in determining the
intensity of stigma. Based on these findings, the present study
hypothesized that race would significantly influence social distance,
perceived danger, blameworthiness, and emotional responses (concern and
help) toward homeless individuals. Specifically, it was predicted that
black homeless individuals would experience greater social distance, be
perceived as more dangerous and more blameworthy, and receive less
concern and help compared to their white counterparts. Additionally, it
was anticipated that individuals with substance use disorders would face
higher levels of social distance, dangerousness, and blame, while
individuals with schizophrenia would receive more concern and help.
Lastly, the study predicted that race and mental illness diagnosis would
interact to influence all aspects of stigmatization.

\subsubsection{Hypotheses}\label{hypotheses}

\paragraph{Effect of Race on
Stigmatization.}\label{effect-of-race-on-stigmatization}

\begin{quote}
\begin{enumerate}
\def\labelenumi{\arabic{enumi}.}
\tightlist
\item
  Black homeless individuals will experience greater social distance.
\item
  Black homeless individuals will be perceived as more dangerous.
\item
  Black homeless individuals will be perceived as more blameworthy.
\item
  Black homeless individuals will receive less concern and help compared
  to white homeless individuals.
\end{enumerate}
\end{quote}

\paragraph{Effect of Mental Illness Diagnosis on
Stigmatization.}\label{effect-of-mental-illness-diagnosis-on-stigmatization}

\begin{quote}
\begin{enumerate}
\def\labelenumi{\arabic{enumi}.}
\tightlist
\item
  Individuals with substance use disorders will face higher levels of
  social distance.
\item
  Individuals with substance use disorders will be perceived as more
  dangerous.
\item
  Individuals with substance use disorders will be perceived as more
  blameworthy.
\item
  Individuals with schizophrenia will receive more concern and help.
\item
  Interaction Between Race and Mental Illness Diagnosis
\end{enumerate}
\end{quote}

\paragraph{Race and mental illness diagnosis
interaction.}\label{race-and-mental-illness-diagnosis-interaction}

\begin{quote}
\begin{enumerate}
\def\labelenumi{\arabic{enumi}.}
\tightlist
\item
  Race and mental Illness diagnosis will interact to influence all
  aspects of stigmatization, including social distance, perceived
  danger, blameworthiness, concern, and willingness to help.
\end{enumerate}
\end{quote}

\section{Method}\label{method}

\subsection{Participants}\label{participants}

The sample for this study consisted of 215 participants, primarily
college-aged students in the United States, with ages ranging from The
ages of the sample ranged from 18 to 79 (M = 35.08, SD = 16.33).The
sample was 44.7\% White, 13\% Hispanic, 33\% Black, 4.2\% Asian
American, 3.3\% Biracial, and 1.9\% other ethnicities. The sample
identified politically as 52.6\% Liberal, 37.1\% Moderate, and 10.3\%
Conservative.The sample was broken down into 1.9\% living in a rural
community, 58.6\% living in the suburbs, 12.1\% living in a small town,
and 27.4\% living in a large metropolitan city. Finally, the
distribution of gender was as follows: 29.3\% male, 68.8\% female, and
1.9\% other responses. \textbf{Table 1.} provides a summary of the
demographic characteristics of the sample.

\subsubsection{Table 1}\label{table-1}

Demographic Information of Participants

\begin{longtable}[]{@{}ll@{}}
\toprule\noalign{}
\textbf{Demographic} & \textbf{Percentage} \\
\midrule\noalign{}
\endhead
\bottomrule\noalign{}
\endlastfoot
\textbf{Age (M = 35.08, SD = 16.33)} & - \\
\textbf{Ethnicity} & \\
White & 44.7\% \\
Hispanic & 13\% \\
Black & 33\% \\
Asian American & 4.2\% \\
Biracial & 3.3\% \\
Other Ethnicities & 1.9\% \\
\textbf{Political Affiliation} & \\
Liberal & 52.6\% \\
Moderate & 37.1\% \\
Conservative & 10.3\% \\
\textbf{Location} & \\
Rural Community & 1.9\% \\
Suburbs & 58.6\% \\
Small Town & 12.1\% \\
Large Metropolitan City & 27.4\% \\
\textbf{Gender} & \\
Male & 29.3\% \\
Female & 68.8\% \\
Other Responses & 1.9\% \\
\end{longtable}

\textbf{Note}: Percentages may not sum to 100 due to rounding.

\subsection{Measures}\label{measures}

This study used multiple questionnaires to examine the effects and
interactions of race and mental illness on stigmatization toward
homeless individuals. Participants were assigned to one of six
experimental conditions using vignettes adapted from Markowitz and
Syverson (\citeproc{ref-markowitzRaceGenderHomelessness2021}{2021}),
manipulating race and mental illness. \textbf{Table 2.} presents the
vignettes used in this study.

\subsubsection{Table 2}\label{table-2}

Vignettes Used in the Study

\begin{longtable}[]{@{}
  >{\raggedright\arraybackslash}p{(\linewidth - 6\tabcolsep) * \real{0.3673}}
  >{\raggedright\arraybackslash}p{(\linewidth - 6\tabcolsep) * \real{0.0816}}
  >{\raggedright\arraybackslash}p{(\linewidth - 6\tabcolsep) * \real{0.1769}}
  >{\raggedright\arraybackslash}p{(\linewidth - 6\tabcolsep) * \real{0.3741}}@{}}
\toprule\noalign{}
\begin{minipage}[b]{\linewidth}\raggedright
\textbf{Condition}
\end{minipage} & \begin{minipage}[b]{\linewidth}\raggedright
\textbf{Race}
\end{minipage} & \begin{minipage}[b]{\linewidth}\raggedright
\textbf{Mental Illness}
\end{minipage} & \begin{minipage}[b]{\linewidth}\raggedright
\textbf{Character Description}
\end{minipage} \\
\midrule\noalign{}
\endhead
\bottomrule\noalign{}
\endlastfoot
Condition 1: Black character/No mental illness & Black & No mental
illness & Male homeless individual with same life story \\
Condition 2: Black character/Substance use disorder & Black & Substance
use disorder & Male homeless individual with same life story \\
Condition 3: Black character/Schizophrenia & Black & Schizophrenia &
Male homeless individual with same life story \\
Condition 4: White character/No mental illness & White & No mental
illness & Male homeless individual with same life story \\
Condition 5: White character/Substance use disorder & White & Substance
use disorder & Male homeless individual with same life story \\
Condition 6: White character/Schizophrenia & White & Schizophrenia &
Male homeless individual with same life story \\
\end{longtable}

\textbf{Note:} All vignettes used the same life story for the male
homeless individual, with only race and mental illness varying across
conditions.

\subsubsection{Attribution
Questionnaire}\label{attribution-questionnaire}

The Attribution
Questionnaire(\citeproc{ref-corriganAttributionModelPublic2003}{P.
Corrigan et al., 2003}) assessed stigmatization aspects like social
distance, blame, perceived dangerousness, emotional response, and
willingness to help.

\subsubsection{Memory Check}\label{memory-check}

A Memory Check assessed participants' recall of story details,
specifically the race and mental illness of the character.

\subsubsection{Demographic
Questionnaire}\label{demographic-questionnaire}

Demographics questionnaire that asked participants about their age,
ethnicity, residence, political affiliation, and familiarity with
homelessness.

\section{Results}\label{results}

\subsection{Descriptive Statistics}\label{descriptive-statistics}

We ran some summary statistics for the sample focusing on it was
inprtant for us to highlight that our sample was composed of mostly a
college age sample despite being open to anyone that wanted to
participate. We highlight this as a limiation to our work.For the age
showed a minimum value of 18, a first quartile of 20, a median of 30, a
mean of 35.08, a third quartile of 50, and a maximum of 79.

In addtion we ran summary statistics for the variable of social dictance
that will be focused in this analysis.For Distance\_mean , the minimum
value was 1, the first quartile was 2.33, the median was 3, the mean was
3.07, the third quartile was 4, and the maximum was 6.67.The results for
this variable are presented in the following section in which an ANOVA
was conduected to examine if there were any significant main effects or
interactions between race and mental illness diagnosis on social
distance stigma.

\subsection{Social Distance Stigma}\label{social-distance-stigma}

The ANOVA results show that the main effect of diagnosis on social
distance stigma showed a significant effect of diagnosis on social
distance stigma, F(1, 211) = 40.31, p \textless{} .001. However, there
was no significant interaction between race and diagnosis on social
distance stigma, F(1, 211) = 0.87, p = 0.512. Similarly, the main effect
of race on social distance stigma did not reach significance, F(1, 211)
= 0.07, p = 0.85. \textbf{Figure~\ref{fig-anova-result}} shows the
F-values for the main effects and interaction.

\begin{figure}[H]

{\caption{{ANOVA F-values for Race, Diagnosis, and Their
Interaction.}{\label{fig-anova-result}}}}

\pandocbounded{\includegraphics[keepaspectratio]{The-Role-of-Race-and-Mental-Illness-Diagnosis-on-Stigmatization-of-Homeless-Individuals_files/figure-pdf/fig-anova-result-1.pdf}}

{\noindent \emph{Note.} The plot shows F-values for each predictor in
the ANOVA model. Higher F-values indicate a stronger effect on the
dependent variable.}

\end{figure}

\subsubsection{Mean Stigma Scores by
Diagnosis}\label{mean-stigma-scores-by-diagnosis}

\begin{figure}[H]

{\caption{{Effect of Mental Illness Diagnosis on Social Distance
Stigma.}{\label{fig-social-distance-stigma}}}}

\pandocbounded{\includegraphics[keepaspectratio]{The-Role-of-Race-and-Mental-Illness-Diagnosis-on-Stigmatization-of-Homeless-Individuals_files/figure-pdf/fig-social-distance-stigma-1.pdf}}

{\noindent \emph{Note.} The plot shows mean social distance stigma
scores by diagnosis, with error bars representing standard deviation.}

\end{figure}

\textbf{Figure~\ref{fig-social-distance-stigma}} shows the effect of
mental illness diagnosis on social distance stigma. The mean stigma
scores for each diagnosis are as follows:No Diagnosis: 2.58 (SD = 1.12),
Schizophrenia: 3.4 (SD = 1.21),Substance Use Disorder: 3.24 (SD = 1.26)

The results indicate that individuals with schizophrenia (M = 3.4) and
substance use disorder (M = 3.24) experience higher levels of social
distance stigma compared to those with no diagnosis (M = 2.58). These
differences were statistically significant.

\subsubsection{Correlations}\label{correlations}

The correlational analysis conducted in this section are to guide where
future research should focus given that there could be other factors
that are influencing the stigmatization of homeless individuals.

A Pearson correlation test was conducted to examine the relationship
between social desirability bias and mean help offering. The results
showed a significant negative correlation between social desirability
bias and mean help offering, r = -0.28, p \textless{} .001, indicating
that participants with higher social desirability bias were less likely
to offer help to homeless
individuals.\textbf{Figure~\ref{fig-correlation-analysis-help-and-social-desirability}}
shows the relationship between social desirability bias and mean help
offering, with a regression line indicating the trend.

\begin{figure}[H]

{\caption{{Correlation between social desirability bias and mean help
offering}{\label{fig-correlation-analysis-help-and-social-desirability}}}}

\pandocbounded{\includegraphics[keepaspectratio]{The-Role-of-Race-and-Mental-Illness-Diagnosis-on-Stigmatization-of-Homeless-Individuals_files/figure-pdf/fig-correlation-analysis-help-and-social-desirability-1.pdf}}

{\noindent \emph{Note.} The plot shows the relationship between social
desirability bias and mean help offering, with a regression line
indicating the trend.}

\end{figure}

\section{Discussion}\label{discussion}

This study explored how race and mental illness diagnosis impact the
stigmatization of homeless individuals, specifically focusing on social
distance. The results revealed that race did not significantly affect
social distance stigma, suggesting that race was not a key factor in
determining social distance in this context. However, the diagnosis of
mental illness had a significant impact on social distance. Individuals
with schizophrenia and substance use disorder were perceived as more
socially distant than those with no diagnosis, supporting the hypothesis
that mental illness contributes to higher levels of social distance
stigma.

The absence of a significant interaction between race and diagnosis
further highlights that mental illness plays a larger role in shaping
social distance perceptions than race. This finding suggests that, when
it comes to social distance, mental illness may be a more prominent
factor than race in the stigmatization of homeless individuals.

These findings underscore the importance of addressing mental illness as
a key determinant in social stigma, particularly in the context of
homelessness. Further research is needed to explore how these factors
interact in other forms of stigmatization and to investigate the role of
mental illness across different populations.

\section{References}\label{references}

\phantomsection\label{refs}
\begin{CSLReferences}{1}{0}
\bibitem[\citeproctext]{ref-corriganPublicStigmaMental2009}
Corrigan, P. W., Kuwabara, S. A., \& O'Shaughnessy, J. (2009). The
{Public Stigma} of {Mental Illness} and {Drug Addiction}: {Findings}
from a {Stratified Random Sample}. \emph{Journal of Social Work},
\emph{9}(2), 139--147. \url{https://doi.org/10.1177/1468017308101818}

\bibitem[\citeproctext]{ref-corriganAttributionModelPublic2003}
Corrigan, P., Markowitz, F. E., Watson, A., Rowan, D., \& Kubiak, M. A.
(2003). An {Attribution Model} of {Public Discrimination Towards
Persons} with {Mental Illness}. \emph{Journal of Health and Social
Behavior}, \emph{44}(2), 162. \url{https://doi.org/10.2307/1519806}

\bibitem[\citeproctext]{ref-frostHomelessnessWasRise2020}
Frost, R. (2020). \emph{Homelessness {Was} on the {Rise}, {Even} before
the {Pandemic} {\textbar} {Joint Center} for {Housing Studies}.}

\bibitem[\citeproctext]{ref-gattisPerceivedRacialSexual2016}
Gattis, M. N., \& Larson, A. (2016). Perceived racial, sexual identity,
and homeless status-related discrimination among {Black} adolescents and
young adults experiencing homelessness: {Relations} with depressive
symptoms and suicidality. \emph{American Journal of Orthopsychiatry},
\emph{86}(1), 79--90. \url{https://doi.org/10.1037/ort0000096}

\bibitem[\citeproctext]{ref-markowitzRaceGenderHomelessness2021}
Markowitz, F. E., \& Syverson, J. (2021). Race, {Gender}, and
{Homelessness Stigma}: {Effects} of {Perceived Blameworthiness} and
{Dangerousness}. \emph{Deviant Behavior}, \emph{42}(7), 919--931.
\url{https://doi.org/10.1080/01639625.2019.1706140}

\bibitem[\citeproctext]{ref-weiszStigmatizedIdentitiesPsychological2018}
Weisz, C., \& Quinn, D. M. (2018). Stigmatized identities, psychological
distress, and physical health: {Intersections} of homelessness and race.
\emph{Stigma and Health}, \emph{3}(3), 229--240.
\url{https://doi.org/10.1037/sah0000093}

\end{CSLReferences}






\end{document}

\documentclass[
  man,
  floatsintext,
  longtable,
  nolmodern,
  notxfonts,
  notimes,
  colorlinks=true,linkcolor=blue,citecolor=blue,urlcolor=blue]{apa7}

\usepackage{amsmath}
\usepackage{amssymb}



\usepackage[bidi=default]{babel}
\babelprovide[main,import]{english}


\babelfont{rm}[,RawFeature={fallback=mainfontfallback}]{Times New Roman}
% get rid of language-specific shorthands (see #6817):
\let\LanguageShortHands\languageshorthands
\def\languageshorthands#1{}

\RequirePackage{longtable}
\RequirePackage{threeparttablex}

\makeatletter
\renewcommand{\paragraph}{\@startsection{paragraph}{4}{\parindent}%
	{0\baselineskip \@plus 0.2ex \@minus 0.2ex}%
	{-.5em}%
	{\normalfont\normalsize\bfseries\typesectitle}}

\renewcommand{\subparagraph}[1]{\@startsection{subparagraph}{5}{0.5em}%
	{0\baselineskip \@plus 0.2ex \@minus 0.2ex}%
	{-\z@\relax}%
	{\normalfont\normalsize\bfseries\itshape\hspace{\parindent}{#1}\textit{\addperi}}{\relax}}
\makeatother




\usepackage{longtable, booktabs, multirow, multicol, colortbl, hhline, caption, array, float, xpatch}
\setcounter{topnumber}{2}
\setcounter{bottomnumber}{2}
\setcounter{totalnumber}{4}
\renewcommand{\topfraction}{0.85}
\renewcommand{\bottomfraction}{0.85}
\renewcommand{\textfraction}{0.15}
\renewcommand{\floatpagefraction}{0.7}

\usepackage{tcolorbox}
\tcbuselibrary{listings,theorems, breakable, skins}
\usepackage{fontawesome5}

\definecolor{quarto-callout-color}{HTML}{909090}
\definecolor{quarto-callout-note-color}{HTML}{0758E5}
\definecolor{quarto-callout-important-color}{HTML}{CC1914}
\definecolor{quarto-callout-warning-color}{HTML}{EB9113}
\definecolor{quarto-callout-tip-color}{HTML}{00A047}
\definecolor{quarto-callout-caution-color}{HTML}{FC5300}
\definecolor{quarto-callout-color-frame}{HTML}{ACACAC}
\definecolor{quarto-callout-note-color-frame}{HTML}{4582EC}
\definecolor{quarto-callout-important-color-frame}{HTML}{D9534F}
\definecolor{quarto-callout-warning-color-frame}{HTML}{F0AD4E}
\definecolor{quarto-callout-tip-color-frame}{HTML}{02B875}
\definecolor{quarto-callout-caution-color-frame}{HTML}{FD7E14}

%\newlength\Oldarrayrulewidth
%\newlength\Oldtabcolsep


\usepackage{hyperref}




\providecommand{\tightlist}{%
  \setlength{\itemsep}{0pt}\setlength{\parskip}{0pt}}
\usepackage{longtable,booktabs,array}
\usepackage{calc} % for calculating minipage widths
% Correct order of tables after \paragraph or \subparagraph
\usepackage{etoolbox}
\makeatletter
\patchcmd\longtable{\par}{\if@noskipsec\mbox{}\fi\par}{}{}
\makeatother
% Allow footnotes in longtable head/foot
\IfFileExists{footnotehyper.sty}{\usepackage{footnotehyper}}{\usepackage{footnote}}
\makesavenoteenv{longtable}

\usepackage{graphicx}
\makeatletter
\newsavebox\pandoc@box
\newcommand*\pandocbounded[1]{% scales image to fit in text height/width
  \sbox\pandoc@box{#1}%
  \Gscale@div\@tempa{\textheight}{\dimexpr\ht\pandoc@box+\dp\pandoc@box\relax}%
  \Gscale@div\@tempb{\linewidth}{\wd\pandoc@box}%
  \ifdim\@tempb\p@<\@tempa\p@\let\@tempa\@tempb\fi% select the smaller of both
  \ifdim\@tempa\p@<\p@\scalebox{\@tempa}{\usebox\pandoc@box}%
  \else\usebox{\pandoc@box}%
  \fi%
}
% Set default figure placement to htbp
\def\fps@figure{htbp}
\makeatother


% definitions for citeproc citations
\NewDocumentCommand\citeproctext{}{}
\NewDocumentCommand\citeproc{mm}{%
  \begingroup\def\citeproctext{#2}\cite{#1}\endgroup}
\makeatletter
 % allow citations to break across lines
 \let\@cite@ofmt\@firstofone
 % avoid brackets around text for \cite:
 \def\@biblabel#1{}
 \def\@cite#1#2{{#1\if@tempswa , #2\fi}}
\makeatother
\newlength{\cslhangindent}
\setlength{\cslhangindent}{1.5em}
\newlength{\csllabelwidth}
\setlength{\csllabelwidth}{3em}
\newenvironment{CSLReferences}[2] % #1 hanging-indent, #2 entry-spacing
 {\begin{list}{}{%
  \setlength{\itemindent}{0pt}
  \setlength{\leftmargin}{0pt}
  \setlength{\parsep}{0pt}
  % turn on hanging indent if param 1 is 1
  \ifodd #1
   \setlength{\leftmargin}{\cslhangindent}
   \setlength{\itemindent}{-1\cslhangindent}
  \fi
  % set entry spacing
  \setlength{\itemsep}{#2\baselineskip}}}
 {\end{list}}
\usepackage{calc}
\newcommand{\CSLBlock}[1]{\hfill\break\parbox[t]{\linewidth}{\strut\ignorespaces#1\strut}}
\newcommand{\CSLLeftMargin}[1]{\parbox[t]{\csllabelwidth}{\strut#1\strut}}
\newcommand{\CSLRightInline}[1]{\parbox[t]{\linewidth - \csllabelwidth}{\strut#1\strut}}
\newcommand{\CSLIndent}[1]{\hspace{\cslhangindent}#1}





\usepackage{fontspec} 

\defaultfontfeatures{Scale=MatchLowercase}
\defaultfontfeatures[\rmfamily]{Ligatures=TeX,Scale=1}

  \setmainfont[,RawFeature={fallback=mainfontfallback}]{Times New Roman}




\title{The Role of Race and Mental Illness Diagnosis on Stigmatization
of Homeless Individuals}


\shorttitle{STIGMA, HOMELESSNESS, MENTAL ILLNESS, AND RACE}


\usepackage{etoolbox}






\author{Karen Veronica Becerra}



\affiliation{
{Department of Psychology, The University of Chicago}}




\leftheader{Becerra}



\abstract{Homelessness in the United States is a persistent problem that
can have serious implications on the well-being of homeless individuals.
The present study focused on the role of race and mental illness
diagnosis on the stigmatization of homeless individuals, specifically
looking at the outcomes of the Attribution Questionnaire. This
questionnaire assessed the aspects of social distance, blame,
dangerousness, concern, and willingness to help of 215 participants
varying in ages across adulthood. The study was a self-paced online form
that used six experimental vignettes. The results indicated that there
were no significant interactions of race x diagnosis on stigmatization.
Additionally, race had no significant main effects, suggesting it was
not a significant factor for stigmatization of homeless individuals.
However, there were some significant main effects of diagnosis. Findings
might suggest that future work in reducing mental illness stigma and
increasing education could help decrease stigmatization of the homeless
population. }

\keywords{Homelessness, Stigmatization, Race, Mental
Illness, Diagnosis, Attribution Questionnaire}

\authornote{\par{\addORCIDlink{Karen Veronica
Becerra}{0009-0006-4967-0955}} 

\par{       }
\par{Correspondence concerning this article should be addressed to Karen
Veronica Becerra, Department of Psychology, The University of
Chicago, 5848 S. University
Avenue, Chicago, IL 60637, USA, Email: kvbecerra@uchicago.edu}
}

\makeatletter
\let\endoldlt\endlongtable
\def\endlongtable{
\hline
\endoldlt
}
\makeatother

\urlstyle{same}



\makeatletter
\@ifpackageloaded{caption}{}{\usepackage{caption}}
\AtBeginDocument{%
\ifdefined\contentsname
  \renewcommand*\contentsname{Table of contents}
\else
  \newcommand\contentsname{Table of contents}
\fi
\ifdefined\listfigurename
  \renewcommand*\listfigurename{List of Figures}
\else
  \newcommand\listfigurename{List of Figures}
\fi
\ifdefined\listtablename
  \renewcommand*\listtablename{List of Tables}
\else
  \newcommand\listtablename{List of Tables}
\fi
\ifdefined\figurename
  \renewcommand*\figurename{Figure}
\else
  \newcommand\figurename{Figure}
\fi
\ifdefined\tablename
  \renewcommand*\tablename{Table}
\else
  \newcommand\tablename{Table}
\fi
}
\@ifpackageloaded{float}{}{\usepackage{float}}
\floatstyle{ruled}
\@ifundefined{c@chapter}{\newfloat{codelisting}{h}{lop}}{\newfloat{codelisting}{h}{lop}[chapter]}
\floatname{codelisting}{Listing}
\newcommand*\listoflistings{\listof{codelisting}{List of Listings}}
\makeatother
\makeatletter
\makeatother
\makeatletter
\@ifpackageloaded{caption}{}{\usepackage{caption}}
\@ifpackageloaded{subcaption}{}{\usepackage{subcaption}}
\makeatother

% From https://tex.stackexchange.com/a/645996/211326
%%% apa7 doesn't want to add appendix section titles in the toc
%%% let's make it do it
\makeatletter
\xpatchcmd{\appendix}
  {\par}
  {\addcontentsline{toc}{section}{\@currentlabelname}\par}
  {}{}
\makeatother

%% Disable longtable counter
%% https://tex.stackexchange.com/a/248395/211326

\usepackage{etoolbox}

\makeatletter
\patchcmd{\LT@caption}
  {\bgroup}
  {\bgroup\global\LTpatch@captiontrue}
  {}{}
\patchcmd{\longtable}
  {\par}
  {\par\global\LTpatch@captionfalse}
  {}{}
\apptocmd{\endlongtable}
  {\ifLTpatch@caption\else\addtocounter{table}{-1}\fi}
  {}{}
\newif\ifLTpatch@caption
\makeatother

\begin{document}

\maketitle


\setcounter{secnumdepth}{-\maxdimen} % remove section numbering

\setlength\LTleft{0pt}


Homelessness in the United States and the struggle to give individuals
adequate housing is a persistent problem. Before the Covid -19 pandemic,
the number of homeless individuals was on the rise with 568,000
individuals experiencing homelessness in 2019, an increase of 15,000
from the previous
year(\citeproc{ref-frostHomelessnessWasRise2020}{Frost, 2020}). With the
current Covid-19 pandemic, we can only predict that those numbers have
continued to increase. In the United States, 2.4\% of homeless
individuals die each year (Stasha, 2020). We know that the general
population often tries to distance itself from the stigmatized
population, more specifically the homeless population. Homeless
individuals face greater stigma and social isolation and often are
removed from public parks and other locations because the general public
does not want them too close. The problems caused by stigmatization,
such as social distancing, can affect the homeless population in terms
of resources that they have available such as sanitation centers,
employment, and social support. Often the homeless population lacks
resources and is exposed to the elements which can increase their
mortality, as well as the chance of being malnourished, having parasitic
infestations, periodontal disease, degenerative joint diseases, venereal
diseases, cirrhosis, and hepatitis-related to intravenous (IV) drug
abuse. Public attitudes toward homeless individuals can influence
policies and the services provided to this population. The attitudes
displayed through the stigma of homeless individuals can have an impact
on both physical and psychological health and willingness to access
services. The impact of these stigmas has shown to have serious
implications on the well-being of homeless individuals. The present
study examined factors that could predict levels of stigmatization
expressed towards homeless individuals.

\subsection{Literature review}\label{literature-review}

\begin{quote}
Research on the stigma of mental illness, homelessness, and race
highlights its harmful effects on health and social integration. P. W.
Corrigan et al. (\citeproc{ref-corriganPublicStigmaMental2009}{2009})
examined public stigma, focusing on stereotypes like causal attribution
(blaming individuals for their condition) and dangerousness (perceiving
them as threatening). Using vignette-based experiments, the study found
that people with psychiatric disorders, especially those with drug
addiction, faced greater stigma than those with physical disabilities.
This work laid the foundation for understanding how mental illness,
particularly schizophrenia and substance use disorders, contributes to
the stigmatization of homeless individuals.

While P. Corrigan et al.
(\citeproc{ref-corriganAttributionModelPublic2003}{2003}) provided
insight into the role of mental illness in stigma, it did not delve
deeply into the specific effects on health outcomes. In contrast,Weisz
and Quinn
(\citeproc{ref-weiszStigmatizedIdentitiesPsychological2018}{2018})
explored the broader implications of homelessness stigma, demonstrating
that it contributes to physical distress, poor health, and social
avoidance. Their study used a sample of 175 volunteers attending a
one-day homeless event, controlling for race, age, mental illness, and
the duration of homelessness. They found that participants who
experienced or anticipated stigma due to homelessness reported higher
psychological distress, worse physical health, and greater reluctance to
use social services. Additionally, participants of color reported even
higher levels of distress, poorer health, and increased avoidance of
services. This research emphasized the intersectionality of homelessness
and race in stigma experiences, offering valuable insights into the
compounded effects of multiple stigmas on individuals' well-being.

Markowitz and Syverson
(\citeproc{ref-markowitzRaceGenderHomelessness2021}{2021}) further
explored the relationship between homelessness and race, examining how
race and gender intersect with stigma. They hypothesized that black
homeless individuals would be perceived as more blameworthy and
dangerous than their white counterparts, and that social distance (i.e.,
the degree of separation individuals feel toward homeless persons) would
be greater for black and male homeless individuals. The study employed a
2 × 2 design, varying race (black vs.~white) and gender (male
vs.~female), and found that black homeless individuals were indeed
perceived as more dangerous, though no significant difference was found
in the level of social distance between black and white individuals.
These findings highlighted the role of race in the perception of
dangerousness, though the effect on blameworthiness was not supported.
The limitations of this study included a sample of college-aged
students, who may have been more tolerant than the general population,
potentially affecting the results. This limitation was addressed in the
present study by including a broader, more diverse participant pool.

(\citeproc{ref-gattisPerceivedRacialSexual2016}{Gattis \& Larson,
2016})focused on racial discrimination and stigma in a sample of 89
black adolescents and young adults who had experienced homelessness. The
study linked racial discrimination and stigma to higher levels of
depression, drawing on the social and minority stress models, which
suggest that marginalized groups experience more psychological distress
due to a lack of societal support. Their findings reinforced the notion
that stigma, particularly racial stigma, contributes to greater
psychological distress in homeless individuals. Although the study was
limited by its small sample size and lack of data on the duration of
homelessness, it underscored the importance of race in shaping the
stigmatization of homeless individuals.

In conclusion, the literature demonstrates that stigmatization due to
homelessness, mental illness, and race significantly impacts
individuals' psychological and physical health, social integration, and
access to resources. The current study seeks to extend this body of
research by examining how mental illness and race interact to influence
stigma levels, providing a more nuanced understanding of the factors
contributing to the stigmatization of homeless individuals
\end{quote}

\subsection{Current Study}\label{current-study}

\begin{quote}
Building on the research by P. W. Corrigan et al.
(\citeproc{ref-corriganPublicStigmaMental2009}{2009}), Markowitz and
Syverson (\citeproc{ref-markowitzRaceGenderHomelessness2021}{2021}), and
Weisz and Quinn
(\citeproc{ref-weiszStigmatizedIdentitiesPsychological2018}{2018}) the
current study aimed to explore how race and mental illness diagnosis
impact the stigmatization of homeless individuals. The research
specifically focused on mental illness, distinguishing between
individuals with schizophrenia and those with substance use disorders,
and examined how these factors interact with race in shaping stigma.
Previous studies suggest that public stigma varies across mental health
conditions and that race plays a crucial role in determining the
intensity of stigma. Based on these findings, the present study
hypothesized that race would significantly influence social distance,
perceived danger, blameworthiness, and emotional responses (concern and
help) toward homeless individuals. Specifically, it was predicted that
black homeless individuals would experience greater social distance, be
perceived as more dangerous and more blameworthy, and receive less
concern and help compared to their white counterparts. Additionally, it
was anticipated that individuals with substance use disorders would face
higher levels of social distance, dangerousness, and blame, while
individuals with schizophrenia would receive more concern and help.
Lastly, the study predicted that race and mental illness diagnosis would
interact to influence all aspects of stigmatization.
\end{quote}

\subsubsection{Hypotheis}\label{hypotheis}

\paragraph{Effect of Race on
Stigmatization.}\label{effect-of-race-on-stigmatization}

\begin{quote}
\begin{enumerate}
\def\labelenumi{\arabic{enumi}.}
\tightlist
\item
  Black homeless individuals will experience greater social distance.
\item
  Black homeless individuals will be perceived as more dangerous.
\item
  Black homeless individuals will be perceived as more blameworthy.
\item
  Black homeless individuals will receive less concern and help compared
  to white homeless individuals.
\end{enumerate}
\end{quote}

\paragraph{Effect of Mental Illness Diagnosis on
Stigmatization.}\label{effect-of-mental-illness-diagnosis-on-stigmatization}

\begin{quote}
\begin{enumerate}
\def\labelenumi{\arabic{enumi}.}
\tightlist
\item
  Individuals with substance use disorders will face higher levels of
  social distance.
\item
  Individuals with substance use disorders will be perceived as more
  dangerous.
\item
  Individuals with substance use disorders will be perceived as more
  blameworthy.
\item
  Individuals with schizophrenia will receive more concern and help.
\item
  Interaction Between Race and Mental Illness Diagnosis
\end{enumerate}
\end{quote}

\paragraph{Race and mental illness diagnosis will interact to influence
all aspects of stigmatization, including social distance, perceived
danger, blameworthiness, concern, and willingness to
help.}\label{race-and-mental-illness-diagnosis-will-interact-to-influence-all-aspects-of-stigmatization-including-social-distance-perceived-danger-blameworthiness-concern-and-willingness-to-help.}

\section{Method}\label{method}

General remarks on method. This paragraph is optional.

Not all papers require each of these sections. Edit them as needed.
Consult the \href{https://apastyle.apa.org/jars}{Journal Article
Reporting Standards} for what is needed for your type of article.

\subsection{Participants}\label{participants}

Who are they? How were they recruited? Report criteria for participant
inclusion and exclusion. Perhaps some basic demographic stats are in
order. A table is a great way to avoid repetition in statistical
reporting.

\subsection{Measures}\label{measures}

This section can also be titled \textbf{Materials} or
\textbf{Apparatus}. Whatever tools, equipment, or measurement devices
used in the study should be described.

\subsubsection{Measure A}\label{measure-a}

Describe Measure A.

\subsubsection{Measure B}\label{measure-b}

Describe Measure B.

\paragraph{Subscale B1.}\label{subscale-b1}

A paragraph after a 4th-level header will appear on the same line as the
header.

\paragraph{Subscale B2.}\label{subscale-b2}

A paragraph after a 4th-level header will appear on the same line as the
header.

\subparagraph{Subscale B2a.}\label{subscale-b2a}

A paragraph after a 5th-level header will appear on the same line as the
header.

\subparagraph{Subscale B2b.}\label{subscale-b2b}

A paragraph after a 5th-level header will appear on the same line as the
header.

\subsection{Procedure}\label{procedure}

What did participants do? How are the data going to be analyzed?

\section{Results}\label{results}

\subsection{Descriptive Statistics}\label{descriptive-statistics}

Describe the basic characteristics of the primary variables. My ideal is
to describe the variables well enough that someone conducting a
meta-analysis can include the study without needing to ask for
additional information.

\section{Discussion}\label{discussion}

Describe results in non-statistical terms.

\subsection{Limitations and Future
Directions}\label{limitations-and-future-directions}

Every study has limitations. Based on this study, some additional steps
might include\ldots{}

\subsection{Conclusion}\label{conclusion}

Describe the main point of the paper.

\section{References}\label{references}

\phantomsection\label{refs}
\begin{CSLReferences}{1}{0}
\bibitem[\citeproctext]{ref-corriganPublicStigmaMental2009}
Corrigan, P. W., Kuwabara, S. A., \& O'Shaughnessy, J. (2009). The
{Public Stigma} of {Mental Illness} and {Drug Addiction}: {Findings}
from a {Stratified Random Sample}. \emph{Journal of Social Work},
\emph{9}(2), 139--147. \url{https://doi.org/10.1177/1468017308101818}

\bibitem[\citeproctext]{ref-corriganAttributionModelPublic2003}
Corrigan, P., Markowitz, F. E., Watson, A., Rowan, D., \& Kubiak, M. A.
(2003). An {Attribution Model} of {Public Discrimination Towards
Persons} with {Mental Illness}. \emph{Journal of Health and Social
Behavior}, \emph{44}(2), 162. \url{https://doi.org/10.2307/1519806}

\bibitem[\citeproctext]{ref-frostHomelessnessWasRise2020}
Frost, R. (2020). \emph{Homelessness {Was} on the {Rise}, {Even} before
the {Pandemic} {\textbar} {Joint Center} for {Housing Studies}.}

\bibitem[\citeproctext]{ref-gattisPerceivedRacialSexual2016}
Gattis, M. N., \& Larson, A. (2016). Perceived racial, sexual identity,
and homeless status-related discrimination among {Black} adolescents and
young adults experiencing homelessness: {Relations} with depressive
symptoms and suicidality. \emph{American Journal of Orthopsychiatry},
\emph{86}(1), 79--90. \url{https://doi.org/10.1037/ort0000096}

\bibitem[\citeproctext]{ref-markowitzRaceGenderHomelessness2021}
Markowitz, F. E., \& Syverson, J. (2021). Race, {Gender}, and
{Homelessness Stigma}: {Effects} of {Perceived Blameworthiness} and
{Dangerousness}. \emph{Deviant Behavior}, \emph{42}(7), 919--931.
\url{https://doi.org/10.1080/01639625.2019.1706140}

\bibitem[\citeproctext]{ref-weiszStigmatizedIdentitiesPsychological2018}
Weisz, C., \& Quinn, D. M. (2018). Stigmatized identities, psychological
distress, and physical health: {Intersections} of homelessness and race.
\emph{Stigma and Health}, \emph{3}(3), 229--240.
\url{https://doi.org/10.1037/sah0000093}

\end{CSLReferences}






\end{document}
